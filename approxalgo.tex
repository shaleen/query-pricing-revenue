\section{Approximation algorithms}
\label{section-approxalgo}

In this section, we present the various approximation algorithms that we consider in our experimental evaluation. We consider algorithms from two subclasses of subadditive and monotone pricing schemes: $(i)$ uniform bundle pricing, and $(ii)$ item (additive) pricing.

\subsection{Uniform Bundle Pricing} 
\label{subsection-uniformbundle}

In uniform bundle pricing, the algorithm sells every hyperedge at a fixed price $P$. Then, if the buyer has valuation $v_e \geq P$, the hyperedge (and thus, the query bundle corresponding to the hyperedge) can be sold. 
To compute the optimal uniform bundle price $P$, we use a folklore algorithm that we call \ubp . The algorithm first sorts the hyperedges in $\mE$
in decreasing order of valuation. Then, it makes a linear pass over the ordered valuations, and for every hyperedge $e \in E$ computes the revenue $R_e$ obtained
if we set the price $P = v_e$. In the end, it outputs $\max_e \{R_e\}$. It is easy to see that the algorithm runs in time $O(m \log m)$, and that it achieves an approximation
ratio of $O(\log m)$.


Uniform bundle pricing is very attractive as a practical pricing scheme, since it has a single parameter (and thus it has a concise representation), and computing the price of a new query is a trivial task. 
However, because it is insensitive to the structure of the bundle (and hence the query), it will perform very poorly when the valuations have a large difference across the hyperedges. 
We should remark here that we expect this to be the case in real-world scenarios: for instance, consider a large table and two queries: one that returns the whole dataset, and one that returns only a single row of the table. Then, it is reasonable to expect that the valuation for these two queries will generally differ by a large margin. 


%Indeed, if all queries are sold at a fixed price, we ignore the actual content of the queries itself \footnote{One can argue that the value of the information content is encoded in the price that customers are willing to pay. This is only true if the buyers know the their valuation function exactly. In general, we want to use pricing functions that are sensitive to the hyperedge structure}. 

%Second, we cannot encode history awareness in the uniform bundle prices. As the buyer purchases more hyperedges over time, we do not want to overcharge the buyer for information that has already been purchased. 
%\shaleen{add an example here}. 
%History awareness is easy to achieve via item pricing but if the pricing function is uniform, we will overcharge the buyer whenever there is overlap between two hyperedges that he/she wants to buy.

%\begin{algorithm}[!htp]
%	\SetCommentSty{textsf}
%	\DontPrintSemicolon 
%	\SetKwFunction{proc}{\textsf{eval}}
%	\SetKw{KwGoTo}{go to}	
%	\SetKwData{maxsum}{maxrev}
%	\SetKwData{fixedprice}{fixedprice}
%	\SetKwData{sum}{rev}
%	\SetKwInOut{Input}{\textsc{input}}\SetKwInOut{Output}{\textsc{output}}
%	%
%	\Input{Hypergraph $H = (V, E)$ and valuation $v_e, e \in E$}
%	\Output{Uniform price $p$}
%	\BlankLine
%	%
%	$\maxsum \leftarrow 0, \fixedprice \leftarrow 0$ \\
%	sort edges in $E$ according to $v_e$\\
%	$n_e \leftarrow \textrm{number of edges }e'\textrm{ with }v_{e'}\geq v_e$
%	\For{$e \in E$}{
%		$\sum \leftarrow v_e * n_e$\\
%		\If{$\sum > \maxsum$}{
%			$\maxsum \leftarrow \sum, \fixedprice \leftarrow p$
%		}
%	}
%
%	\KwRet{$\fixedprice$}
%	\caption{Find revenue maximizing uniform bundle price}
%	\label{algo:uniformbundleprice}
%\end{algorithm}

%Algorithm~\ref{algo:uniformbundleprice} shows how to find the revenue maximizing uniform bundle price which gives an $O(\log |E|)$ approximation %where $v_{\max}, v_{\min}$ are the largest and smallest valuations for any hyperedge respectively.

\subsection{Item Pricing} 

In item pricing, we seek to assign a weight $w_j \geq 0$ to each vertex in the hypergraph. 
Then, the price of any hyperedge $e$ is given by $p(e) = \sum_{j \in e} w_j$. The representation size of item pricing is $O(n)$, so we can guarantee that it will have a 
concise representation as long as we pick the support set to be small enough (recall that $n = |\mS|$). Unlike uniform bundle pricing, item pricing can capture large differences between
the valuation of different queries. On the other hand, we also need to choose a large enough support size, so that we can extract a reasonably good revenue from our pricing
function. A large support size also guarantees that that new queries that arrive will have
non-empty hyperedges and hence will not be priced to 0. 

%Item pricing is a natural fit for the query pricing framework since it allows for maintaining history of query purchases and is sensitive to the information content of the query. However, unlike uniform bundle pricing, we need to ensure that $\mS$ is big enough so that queries have a non-zero hyperedge size. If any query has no disagreement, then item pricing will assign a zero price to the query. 

In our experiments, we evaluate four item pricing algorithms.

\paragraph{The Uniform Item Pricing (\uip) Algorithm.}
Our first item pricing algorithm is an $O(\log n + \log m)$-approximation algorithm given by Guruswami et. al~\cite{guruswami2005profit}. \uip\ outputs a uniform item pricing,
where every $w_j$ is set to the same value $w$. The algorithm sorts all hyperedges on the value $q_e = \frac{v_e}{|e|}$, computes for every hyperedge $e$  
the revenue $R_e$ that we can obtain if we set $w_j = q_e$ for every $j \in \mV$, and finally outputs the pricing that gives $\max_e \{ R_e\}$. Its running time is also $O(m \log m)$.


\paragraph{The LP Item Pricing (\lpip) Algorithm.}

The second item pricing algorithm we consider builds upon the \uip\ algorithm to construct a non-uniform item pricing. \lpip\ constructs a separate linear program $LP(e)$ for every hyperedge $e \in \mE$ as follows. Let $F_e \subseteq \mE$ be the set of hyperedges $e'$ such that $v_{e'} \geq v_e$. Then, $LP(e)$ has the objective of maximizing the revenue,
with the constraint that every edge in $F_e$ must be sold: in other words, for every $e' \in F_e$, we must have $\sum_{j \in e'} w_j \leq v_{e'}$. Observe that the uniform item pricing
solution that sets each weight $w_j$ to $v_e$ is a feasible solution for $LP(e)$, so the output of the linear program can only give a better item pricing. \lpip\ outputs the revenue-maximizing solution across all $LP(e)$. The worst-case approximation guarantee of \lpip\ is also $O(\log m)$; however, as we will see in the experimental section, it often outperforms \uip.

\paragraph{The Capacity Item Pricing (\cip) Algorithm.}
This algorithm is an $O(\log B)$-approximation algorithm given by ~\cite{cheung2008approximation}.
Although this primal-dual algorithm was presented in the context of item pricing with limited supply, it readily extends to the unlimited supply setting. Intuitively, \cip\ sets a uniform capacity constraint $k$ on how many times each item (vertex) can be sold, and for that $k$ solves a linear program that optimizes for the welfare-maximization problem.  The dual solution of this LP gives the prices of items such that at least $k$ copies of each item are sold. \cite{cheung2008approximation} proves that if we search through the possible capacity constraint using a step-size of $(1+\epsilon)$ -- so $k = 1, (1+\epsilon), (1+\epsilon)^2, \dots$ -- then the revenue-maximizing item pricing across all $k$'s achieves an approximation ratio of $O((1+\epsilon) \log B)$.
 
%\begin{algorithm}[!htp]
%	\SetCommentSty{textsf}
%	\DontPrintSemicolon 
%	\SetKwFunction{proc}{\textsf{eval}}
%	\SetKw{KwGoTo}{go to}	
%	\SetKwData{maxsum}{maxrev}
%	\SetKwData{fixedprice}{fixedprice}
%	\SetKwData{temp}{temp}	
%	\SetKwInOut{Input}{\textsc{input}}\SetKwInOut{Output}{\textsc{output}}
%	%
%	\Input{Hypergraph $H = (V, E)$ and valuation $v_e, e \in E$}
%	\Output{Item weights $o$}
%	\BlankLine
%	%
%	$\maxsum \leftarrow 0, \fixedprice \leftarrow 0, t$ \\
%	\For{$e \in E$}{
%		$q \leftarrow \frac{v_e}{|E|}, E' = \emptyset, w_v = \langle 0, \dots, 0 \rangle$ \\
%		\For{$e \in E$}{
%			\If{$v_e \geq |E| \cdot q$}{
%				$E' \leftarrow E' \cup e$ 
%			}
%		}
%		
%		$\temp \leftarrow $maximize  $\sum_{e \in E'} w_v \cdot M[e]$ \\
%		subject to \\
%		$w_v \cdot M[e] \leq w_e$ for all $e \in E'$ \\
%		$w_v \geq 0$
%		
%		\If{$\temp > \maxsum$}{$\maxsum \leftarrow \temp, t \leftarrow w_v$}
%	}
%	\KwRet{$t$}
%	\caption{$O(\log B)$-approximation item pricing algorithm}
%	\label{algo:itempricingbase}
%\end{algorithm}

\paragraph{The Layering Algorithm}
Since the previous algorithms require solving multiple linear programs, they can be slow when the size of input is large.  We thus also
consider a fast greedy algorithm that achieves an $O(B)$-approximation in the worst case (but as we will see, a much better approximation
in practice).

\begin{algorithm}[!htp]
	\SetCommentSty{textsf}
	\DontPrintSemicolon 
	\SetKwFunction{proc}{\textsf{eval}}
	\SetKw{KwGoTo}{go to}	
	\SetKwData{maxsum}{maxrev}
	\SetKwData{fixedprice}{fixedprice}
	\SetKwData{temp}{temp}	
	\SetKwInOut{Input}{\textsc{input}}\SetKwInOut{Output}{\textsc{output}}
	%
	\Input{Hypergraph $\mH = (\mV, \mE)$ and valuation $\{ v_e\}_{e \in \mE}$}
	\Output{Item pricing $w_j$ for each item $j \in \mV$}
	\BlankLine
	%
	$Rev \leftarrow 0$, $S\leftarrow \emptyset$, $w_j \leftarrow 0$ for each $j \in \mV$;\\
	\While{$\mE \neq \emptyset$}{
		Let $\mE'\subseteq \mE$ be a minimal set cover of the items in $\bigcup_{e \in \mE} e$;\\
		\If{$\sum_{e\in \mE'}v_e>Rev$}{
			$S\leftarrow \mE'$;\\
			$Rev\leftarrow \sum_{e\in \mE'}v_e$;\\
		}
		$\mE\leftarrow \mE \setminus \mE'$;
	}
	\For{$e\in S$}{
		Find item $j \in e$ such that $j \not\in e'$, $\forall e'\in S$, $e'\neq e$;\\
		$w_j \leftarrow v_e$; 
	}
	\KwRet{$\{w_j\}_{j \in \mV}$}
	\caption{The Layering Algorithm}
	\label{algo:layering}
\end{algorithm}

The following theorem proves the correctness of Algorithm \ref{algo:layering}, and analyzes its performance.

\begin{theorem}
\label{thm-Bapprox}

Algorithm~\ref{algo:layering} outputs a $B$-approximation item pricing in $O(Bm)$ time. 

\end{theorem}

\begin{proof}
Each step the algorithm finds a minimal set cover of the remaining items, call the set cover a \textit{layer}.
On one hand, since each item presents in at most $B$ hyperedges, there are at most $B$ layers as at each step the degree of each item decreases by at least 1. On the other hand, each hyperedge $e$ in a minimal
set cover $\mE'$ must contain at least one unique item that is not contained in other sets: otherwise $\mE'\setminus \{e\}$ is still a set cover,
which contradicts the minimality of $\mE'$. Pricing these unique items at price equal to the value of corresponding sets can extract full revenue from the hyperedges in this layer. There must exist a layer such that the item pricing can achieve 
$\Omega(\frac{1}{B})$ of the total value of all the hyperedges, thus this item pricing algorithm has approximation ratio $O(B)$. The running time for each step is $O(m)$, and the total running time is $O(Bm)$.
\end{proof}
