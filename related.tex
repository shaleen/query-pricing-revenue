\section{Related Work}
\label{sec:related}


\paragraph{Query-Based Pricing.}
There exist various simple mechanisms for data pricing (see~\cite{muschalle2012pricing} for a survey on the subject), including a flat fee tariff, usage-based and output-based prices. 
These pricing schemes do not provide any guarantees against arbitrage.
%
The vision for arbitrage-free query-based pricing was first introduced by Balazinska et al.~\cite{balazinska:11b}, and was further developed in a series of papers~\cite{KUBHS12,KUBHS13,KUBHS12b}. The proposed framework requires that the seller sets fine-grained price points, which are prices assigned to a specific type of queries
over the dataset; these price points are used as a guide to price the incoming queries. 
Even though the pricing problem in this setting is in general NP-hard, the QueryMarket prototype~\cite{KUBHS13} showed that it is feasible to compute the prices for small datasets, albeit not in real-time. 

%In \textsc{Qirana}, price points are not a first-class citizen; we can compute fine-grained query prices even when the seller has specified a single price for the whole dataset, by assigning uniform value to every piece of the data. The seller provides price points only when necessary. In contrast, QueryMarket requires the seller to specify very fine-grained price points; even if the seller decides to assign the same equal price to every available price point, this can lead to undesired effects where a particular attribute value that touches most of the data is valued the same as one that touches a small part of the data. Another drawback of QueryMarket w.r.t. the price points is that it is not possible to control the price on an attribute-level, something that \textsc{Qirana} supports.

Further work on data pricing proposed new criteria for interactive pricing in data markets~\cite{LM12}, and described new necessary arbitrage conditions along with several negative results related to the trade-off between flexible pricing and arbitrage avoidance~\cite{LK14}. Upadhyaya et al.~\cite{Upadhyaya2016} investigated history-aware pricing using refunds. More recently, Deep and Koutris~\cite{deep2016design} characterized the possible space of pricing functions with respect to different arbitrage conditions. The theoretical framework was then implemented as part of the \textsc{Qirana} system~\cite{deep2017qirana}, which can support pricing in real-time. The framework we use in this paper for query-based pricing is the same one from~\cite{deep2017qirana}. 


\paragraph{Revenue Maximization.}
Revenue-maximizing mechanisms have been well understood in single-item auctions, where the posted pricing
mechanism is optimal \cite{myerson1981optimal}. 
However in general multi-parameter settings, revenue-maximizing mechanisms are considered
hard to characterize. In the past few decades many researchers started to 
focus on simple and approximately optimal solutions, especially posted-pricing mechanisms.
Recently a line of work shows that in Bayesian setting with limited supply, posted pricing achieves 
constant approximation when there is single buyer 
\cite{babaioff2014simple, chawla2007algorithmic, 
chawla2010multi, chawla2015power, rubinstein2015simple},
and logarithmic approximation (with respect to the number of items) when there are multiple buyers 
\cite{cai2016duality, chawla2016mechanism, caizhao2017duality}.

The case that our paper focuses on has all valuations of buyers revealed to the seller. This setting
is initiated by \cite{guruswami2005profit}, which shows that item pricing 
gives $O(\log n)$-approximation for unit-demand buyers
in limited-supply setting, and $O(\log n+\log m)$-approximation for single-minded buyers in unlimited supply
setting. The competitive ratio for unlimited supply setting was improved to $O(\log k + \log B)$ 
by \cite{briest2006single} then to $O(\log B)$ by \cite{cheung2008approximation} where 
$k$ denotes the size of largest bundle, and $B$ denotes the maximum number of bundles containing a specific item.
Another line of work studies how to find best possible item pricing in above setting where $k$ is bounded. Such problem 
is also known as the \textit{k-hypergraph pricing} problem.
\cite{briest2006single} first gives a polynomial-time algorithm finding an approximately optimal item pricing
with competitive ratio $k^2$. The approximation ratio is improved to $k$ by \cite{balcan2006approximation},
which is proven to be near-optimal: under the Exponential Time Hypothesis there is no polynomial-time algorithm 
that achieves competitive ratio $k^{1-\epsilon}$ \cite{chalermsook2013independent}.

To the best of our knowledge, the empirical performance of these algorithms has not been studied previously, whether on real or synthetic data.

\paragraph{Pricing information.}
Another line of research in economics considers the revenue maximization problem for a seller offering to sell information. See, for example, \cite{BKP-12, BB-15, BBS-17} and references therein. However, that literature differs from our work in several fundamental aspects. First, in those works, both the seller and the buyer are unaware of the true state of the information (\ie, the dataset), and this state is stochastic. Second, the seller is allowed to sell queries whose results are randomized. Third, the buyer's type (which information he is interested in and his value) are unknown to the seller. In contrast, in our setting while the pricing is required to be arbitrage-free, the types of the buyers are known to the seller in advance. As such the two models lead to very different types of pricing mechanisms and algorithms.