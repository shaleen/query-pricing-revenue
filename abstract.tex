\begin{abstract}

In this paper we study the problem faced by a broker selling access to data who wants to maximize his revenue. The broker can set different prices for different queries made to the dataset, but must ensure that the pricing function does not provide the buyers with opportunities for arbitrage. This problem can be formulated as a revenue maximization problem with single-minded buyers and unlimited supply, for which several approximation algorithms are known. In this paper, we perform an extensive empirical evaluation of the performance of several pricing algorithms for the query pricing problem on industry scale workloads. In addition to previously known approximation algorithms, we propose several new heuristics and analyze them both theoretically and experimentally. Our experiments show that algorithms with the best theoretical bounds are not necessarily the best empirically. We identify algorithms and heuristics that are fast as well as provide consistently good performance when valuations are drawn randomly from a variety of distributions including zipfian, normal, and uniform distributions.
% In this paper, we present approximation algorithms for {\em query pricing} problem so as to maximize seller's revenue in the unlimited supply setting. We formulate the problem as an instance item pricing over {\em item hypergraph} where the buyers are {\em single minded} with subadditive valuations and investigate a variety of pricing strategies including {\em uniform item pricing} and {\em uniform bundle pricing}. \cite{guruswami2005profit} gave an $O(\log m + \log n)$ where $m$ is the number of customers and $n$ is the number of items with sum of all valuations as the upper bound. In our setting, we show that both item pricing and uniform bundle pricing exhibit a gap of $\Omega(m)$ from the optimal subadditive bundle pricing. Additionally, we investigate the revenue obtained using XOS pricing functions and show $O(\dots)$ approximation with a matching lower bound. We complement our theoretical results with extensive experiments on industry scale workloads. To this end, our experiments show that pricing algorithms behave very well in practice when valuations are drawn randomly from a variety of distributions including zipfian, normal and uniform distribution.
\end{abstract}
