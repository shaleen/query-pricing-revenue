\section{The Query-Based Pricing Framework}
\label{sec:framework}

In this section, we present the framework of query-based pricing, and then formally describe the pricing problems we tackle.

\subsection{Query-Based Pricing Basics}

The {\em data seller} wants to sell a database instance $\db$ through a data market, which functions as the broker. The instance has a fixed relational schema $\bR = (R_1, \dots, R_k)$ with $k$ relations. We denote by $\mI$ the set of possible database instances. The set $\mI$ encodes information about the data that is provided by the data seller, and is public information known to any buyer (together with the schema). We allow the set $\mI$ to be infinite, but countable. For example, suppose that the schema consists of a single binary relation $R(A,B)$, and the domain of both attributes is $[n] = \set{1, \dots, n}$. Then, $\mI = 2^{[n] \times [n]}$, \ie the set of all directed graphs on the vertex set $[n]$.

{\em Data buyers} can purchase information from the dataset by issuing queries in the form of a {\em query bundle}  $\bQ=\qb{Q_1, \hdots, Q_n}$, which is a vector of queries. For our purposes, a query $Q$ is a deterministic function that takes as input a database instance $\db$ and returns an output $Q(\db)$.  
We denote the output of the query bundle by $\bQ(\db) = \qb{Q_1(\db), \hdots Q_n(\db)}$. 

A {\em pricing function} $p(\bQ, \db)$ takes as input a query bundle $\bQ$ and a database instance $\db \in \mI$ and assigns to it a price, which is a number in $ \mathbb{R}_+$. 
The reason we consider query bundles in our setting is that in practice a data buyer will issue over time a sequence $\bQ_1, \dots, \bQ_m$ of query bundles on the database. In this case, after issuing the first $i$ queries, the data buyer should not be charged a price of $\sum_i p(\bQ_i,D)$, but instead $p(\bQ_1, \dots, \bQ_i, D)$. 

Assigning prices to query bundles without any restrictions can lead to arbitrage opportunities. There are two different conditions where arbitrage may occur:

\introparagraph{Information Arbitrage} The first condition captures the intuition that if a query bundle $\bQ_1$ reveals a subset of information of what a query bundle $\bQ_2$ reveals, then the price of $\bQ_1$ must be less than the price of $\bQ_2$. If this condition is not satisfied, it creates an arbitrage opportunity, since a data buyer can purchase $\bQ_2$ instead, and use it to obtain the answer of $\bQ_1$ for a cheaper price. 

Formally, we say that $\bQ_2$ determines $\bQ_1$ under database $\db$, denoted $\db \vdash \bQ_2 \dtr \bQ_1$ if for every database $\db' \in \mI$ such that $\bQ_2(\db) = \bQ_2(\db')$, we also have  $\bQ_1(\db') = \bQ_1(\db)$.
We say that the pricing function $p$ has no {\em information arbitrage} if for every database $\db \in \mI$ such that $\db \vdash \bQ_2 \dtr \bQ_1$, we have $p(\bQ_2, \db) \geq  p(\bQ_1, \db)$. \todo{example}

\introparagraph{Bundle Arbitrage} The second condition regards the scenario where a data buyer wants to obtain the answer for the query bundle $\bQ = \bQ_1\Vert \bQ_2$, where $\Vert$ denotes vector concatenation. Instead of asking $\bQ$ as one, the buyer can create two separate accounts, and use one to ask for $\bQ_1$ and the other to ask for $\bQ_2$. To avoid such an arbitrage situation, we must make sure that the price of $\bQ$ is at most the sum of the prices for $\bQ_1$ and  $\bQ_2$. \todo{example}
%
Formally, we say that the price function $p$ has no {\em bundle arbitrage} if for every database $\db \in \mI$, we have $p(\bQ_1\Vert \bQ_2, \db) \leq p(\bQ_1, \db) + p(\bQ_2, \db)$.

We say that the pricing function $p$ is {\em arbitrage-free} if it has no information arbitrage and no bundle arbitrage. 

\subsection{From Pricing Queries to Pricing Bundles}

In general, computing whether $\bQ_2$ determines $\bQ_1$ under some database $\db$ is an intractable problem. To overcome this obstacle, we take a different view of a query bundle. Let 
$\mS \subseteq \mI$ be any subset of $\mI$, called the {\em support set}, and define the {\em conflict set} of $\bQ$ w.r.t. to $\mS$ as:
%
\begin{align*}
\dagr{\mS}{\bQ,\db}  &= \setof{\db' \in \mS}{\bQ(\db) \neq \bQ(\db')}. 
\end{align*}

Intuitively, the conflict set contains all the database instances from $\mS$ for which the buyer knows that cannot be the underlying instance $\db$ once she learns the answer $\bQ(\db)$. This construction maps each query to the {\em  bundle} $\dagr{\mS}{\bQ,\db} $ over the common itemset $\mS$. We should remark here that the task of computing the bundle $\dagr{\mS}{\bQ,\db} $ is computationally feasible if we choose $\mS$ to be small enough, since we can simply iterate through all the items $\db' \in \mS$, and for each item check the condition 
$\bQ(\db) \neq \bQ(\db')$.

We can now compute a price for $\bQ$ by applying a set function $f: 2^{\mS} \rightarrow  \mathbb{R}_+$ on $\dagr{\mS}{\bQ, \db}$. 
\hlrone{A set function $f$ is {\em monotone} if for sets $A \subseteq B$ we always have $f(A) \leq f(B)$, and {\em subadditive} if for every set $A,B$ we have $f(A) + f(B) \geq f(A \cup B)$.}
By choosing $f$ to be monotone and subadditive, we can guarantee that the pricing function is arbitrage-free.

\begin{theorem}[\cite{deep2016design}] \label{cor:arbitrage}
Let $\mS \subseteq \mI$, and $f$ be a monotone and subadditive set function $f: 2^{\mS} \rightarrow  \mathbb{R}_+$. Then, the function $p(\bQ, \db) = f(\dagr{\mS}{\bQ, \db})$ is arbitrage-free.
\end{theorem}

\subsection{Revenue Maximization}

We consider the unlimited supply setting, where the data seller can sell any number of units of each query. 
Additionally, we assume that the buyers are single-minded: each buyer is interested in buying only a single query bundle $\bQ$; the buyer will purchase $\bQ$ only if the price $p(\bQ, \db) \leq v_\bQ$, where $v_\bQ$ is the valuation  that the buyer has for $\bQ$. 

The problem setup is as follows. We are given as input a set of $m$ buyers, where each buyer $i$ is interested of purchasing a query bundle $\bQ_i$ with valuation $v_i$. We pick a support set $\mS \subseteq \mI$ of size $n = |\mS|$. By using the transformation of queries to bundles over $\mS$, we can construct a hypergraph $\mH = (\mV,\mE)$, with vertex set $\mV = \mS$, and hyperedges $\mE = \setof{e_i}{i =1, \dots, m}$, where $e_i =\dagr{\mS}{\bQ_i, \db}$.
The task in hand is to find a set function $p$\footnote{Here we have overloaded $p$ to also be a pricing function with input a bundle of items.} that maximizes the total revenue $\sum_{i=1}^m p(e_i)$, with the restriction that $f$ must be both monotone and subadditive. \paris{should we require that $f$ is a total function?}

For practical applications (e.g., \textsc{Qirana}~\cite{deep2017qirana}), we must only consider functions that can be both concisely represented and also efficiently computable. For example, it is not desirable to come up with a function $p$ where we need to explicitly store all the $2^n$ values for all input bundles from $\mS$.
For this reason, we focus on a few important subclasses of monotone and subadditive set functions:
%
\begin{packed_item}
\item The {\em uniform bundle price}  $p^b(\cdot)$ assigns the same price to every hyperedge, \ie $p^b(e) = P$ for some number $P \geq 0$.
%
\item The {\em additive price} $p^a(\cdot)$ assigns a weight $w_j \geq 0$ to every item $j \in \mS$, and then defines
$p^a(e) = \sum_{j \in e} w_j$. Such a pricing function is also commonly known as an {\em item pricing}~\cite{}.
%
\item The {\em XOS price} $p^x(\cdot)$ defines $k$ weights $w_j^1, w_j^2, \dots, w_j^k$ for each item $j \in \mS$, and then sets the price to $p^x(e) = \max_{i=1}^k \sum_{j \in e} w_j^i$.
\end{packed_item}

Given the above three subclasses of pricing functions, we consider the following two questions, both from a theoretical and practical point of view.
First, how much do we lose in terms of revenue by replacing the optimal monotone and subadditive pricing function with a uniform, additive or XOS pricing function? In other words, we seek to understand what is the revenue we lose for the sake of computational efficiency.
Second, we want to develop algorithms that can optimize the prices for each subclass and achieve a good approximation ratio with respect to the optimal revenue.


